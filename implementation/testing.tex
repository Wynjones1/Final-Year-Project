Testing of the program was performed in two main ways, testing of the indidual functions and testing of the system as a whole.
certain functions designed to accelerate the raytracing need to be compared to a base implementation that while slower is
guaranteed to be correct, for instance a finding the closest intersection of a ray and mesh, the simplest way to do this
is to test the ray and every triangle in the mesh and record the closest intersection, this is simple to perform and
the scope for errors in the implementation is low, unfortuanatly this is a very expensive operation that requires $\Omega(n)$
time complexity, in this project the k-d tree is used to accelerate the test, this can be performed in $\Omega(log(n))$ time
complexity, the code to perform the intersection test for this data structure is much more complicated and as a result
errors in the implementation are more likely. Testing of these complicated functions was done by comparing the output
of the simple function with the faster implementation this can be used to ensure that the outputs match, it can also be
used to see if the new implementation does indeed increase the performance of the code. Figure~\ref{fig:testing_performance_comp}

\begin{figure}[h!]
\missingfigure{Performance Comparison}
\label{fig:testing_performance_comp}
\caption{Performance difference for triangle intersections}
\end{figure}
