\section{Photon mapping advances}
Since the initial creation of the photon mapping algorithm in 1996 there have been many improvements
and modifications to the photon mapping algorithm in order to increase the efficiency and the
range of effects that can be visulised.

\subsection{Reverse Photon Mapping}
In order to speed up the photon mapping algorithm Havran et al. developed the reverse photon mapping
algorithm, it is claimed in his paper that substantial speedups are possible by performing the
raytracing stage before the photon mapping stage by accessing memory in a more coherent manner.

\subsection{Progressive Photon Mapping}
Progressive Photon Mapping \cite{Hachisuka08} is a reformulation of the photon mapping algorithm
into a multipass solution to the rendering equation in which the raytracing stage is performed prior
to the photon mapping stage in a simalar way to reverse photon mapping.

In the traditional photon mapping algorithm the number of photons that are emitted into the scene
are fixed, as such it is not possible to obtain an estimate of the radiance to an arbitrary
precision, this can cause the estimate of the radiance to be blured. Progressive photon mapping
attempts to solve this problem by performing more than one photon mapping pass. After each photon
map stage the search radius at each pixel is reduced, this allows details to be refined through
subsequent passes. In addition to the ability to refine the image quality the memory requirement to
produce an image to a given precision is reduced as each photon map does not need to be stored in
main memory.

While this algorithm has been shown to be a useful addition to the traditional photon mapping
algorithm it is much more complicated, we need to perform more than one photon mapping stage which
require the radiance estimate to be recalculated multiple times.

\subsection{Stochastic PPM}
Stochatic Progressive Photon Mapping (SPPM) \cite{Hachisuka09} is a recently extension to PPM that
is simalar in motivations as distributed raytracing, that is to be able to use the algorithm to
produce phenomena such as depth of field, motion blur and glossy reflections. This method has been
shown to converge to the correct solution for these phenomena and specular-diffuse-specular paths
faster that normal PPM.

\subsection{Dynamic Scenes}
Although the photon mapping algorithm has traditionaly been used for static scenes there has
recently been work towards modifiing the algorithm so that it can be used for dynamic scenes, the
work of Weiss et al. \cite{Weiss12} in this paper it presents a method of rendering that uses
information from previous frames of the scene in order to gain a speedup of the rendering process.
