\section{The BDRF}
\label{sec:bdrf}
Most global illumination systems need to describe the radiance from an object, this includes the
reflected radiance by an object from the scene.
The bidirectinoal reflectance distribution function \cite{Nicodemus65}is a function that describes 
the reflectance of a surface at point $x$ with respect to an outwards direction $\omega_{o}$, and 
inwards direction $\omega_{i}$. The BDRF is commonly written as $f_{r}(x, \omega_{i},\omega_{o})$. 
The BDRF is a vital part of computer graphics as it describes the reflectance of a surface which is 
vital in global illumination as we need to consider reflectance from all objects in a scene.

The BRDF is a mesure of the outgoing radiance $L$ in a given direction $\omega_{o}$ from an incoming
irradiance $E$ from a direction $\omega_{i}$. The BRDF for a surface is given by:

\begin{equation}
f_{r}(\omega_{i}, \omega_{o}) = \frac{L_{\omega_{i}}}{E_{\omega{o}}}
\end{equation}

An example of a simple BDRF would be that of Lambertian reflection whereby the light is reflected
in all directions equally, the BRDF for this is:

\begin{equation}
f_{r}(\omega_{i}, \omega_{o}) = \frac{\rho}{\pi}
\end{equation}

Where $\rho$ is the reflectivity of the surface, for physically based rendering this is a real
number between zero and one.

For the system that I am developing it is important to be able to specify the material of an
object in the scene in a flexible manner, this will include a method of representing the BDRF of
an object that can be easily configured.

\section{The Rendering Equation}
Central to most modern rendering systems is the rendering equation, the rendering equation is an
approximation to the light transfer of a scene \cite{Kajiya86}. A frequency independant version of
the rendering equation is given below.

\begin{equation}
L_{r}(x, \omega_{o}) = L_{e}(x, \omega_{o})
					 + f(x)
					   \int_{\Omega}
					 		f_{r}(x, \omega_{i}, \omega_{o})
							(\omega_{i} \cdot n)d\omega_{i}
\end{equation}
Where $L_{r}$ is the radiance from the surface at point x, $L_{e}$ the emmited radiance from the
surface and $f_{r}$ is the BDRF as described in section~\ref{sec:bdrf}.

There have been many developments within computer graphics that attempt to find approximations
to the rendering equation, radiosity \cite{Goral85}which attempts to find the solution by splitting 
the geometry of a scene into smaller patches and builds a system of linear equations that solve the 
radiosity value for the patch, this technique is simple but sufferes from some problems, for one it 
is only able to account for lambertian diffuse reflections from other object and so cannot be used
to render specular reflections. Radiosity is still used within architectural applications as the
radiometry is view independant and as such is well suited to applications such as walk-throughs.
Monte-carlo methods are another method that is more general that radiometry, this method is a
stocastic method that attempts to find the solution of the rendering equation by sampling the light
transfer of a point untill a adequete approximation is found, a common form of this is distributed
raytracing whereby raytracing as introduced by Whitted \cite{whitted79a} is performed but for each
intersection the choice of reflecting, refracting or absoring is taken from a distribution as the
number of rays emmited increases the image converges to the solution of the rendering equation.

Understandind the motives of the rendering equation will be vital if I am to be sucsessfull in
developing a photon mapping system, the solution that most rendering algorithms produce are an
approximation to the rendering equation and as such will be the focus of my system.
