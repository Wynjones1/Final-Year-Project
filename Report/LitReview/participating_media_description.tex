\section{Participating media.}
Participating media are materials or objects that interact with light within the surface of the
object, these interactions can create interesting visual effects that add to the realism of a 
scene, a good example of participating media would be smoke, as light enters smoke it will
experience scattering that will cause the path of the light to be altered. \cite{Blinn82}
In order to render participating media a solution to the volume rendering equation must be found,
this equation describes the light transfer within the participating media, Jensen presents a
good explination \cite{JensenBook}.

Participating media are generally split into two types, inhomogenios and homogenios, this describes
if the properties of the participating media are constant, an example of a
participating media that is homogenios would be a liquid such as milk, an example of a
non-homogenios media woulf be smoke as the density of the smoke (and as a result how much it scatters
light) is not constant.

Early work in this area include work done by Blinn \cite{Blinn82}, in this paper a model for the
scattering in clouds and other such media is given, this paper only considers single scattering
in its approximation and as noted in \cite{Jensen98} only suitible for optically thin media.

\subsection{Volume ray casting}
Volume ray casting is a technique that is central to participating media as it allows rays to be
seen as moving through a media, this is apposed to traditional raytracing where the ray will only
interact at the point of intersection of an object. In ray casting the ray is "marched" through
the medium, at each step in the process a calculation can be performed in order to decide if an
action should be taken (i.e in or out scattering)

\subsection{Volumetric Photon Mapping}

One of the earliest applications of photon mapping to render participating media was \cite{Jensen98}
in this paper the concept of volumetric photon mapping and the volume photon map was introduced,
this photon map allows the interaction of light within a participating media to be taken into
account, this could be one of the following, in-scattering, out-scattering, absorbsion or emmission.
The volume map is a seperate map to the surface map usually used in the photon mapping algorithm,
this is because the desity estimate is different as the sample of photons are taken from the volume
enclosing $n$ photons and not the disk. The technique in this paper deals with the case of 
homgenious and non-homogenious media as the volume map decouples the representation of the radiance 
from the geometry of the media.

Building the volumetric photon map is different to that of the surface volum map as the photons
can interact anywhere within the definition of the volume, there is defined for the volumne a
cumulatice probablility density function that defines the probability of an interaction at all
points $x$ within the volume.

\begin{equation}
F(x) = 1 - \tau(x_{s}, s)
\end{equation}

where $\tau$ is the transmittance which is computed by ray marching.

\begin{equation}
\label{eq:radiance_volumn_estimate}
L_{r}(x, \vec{\omega})
\approx
\frac
{1}
{\sigma(x)}
\sum_{p=1}^n
f_{r}(x,\vec{\omega}_{p}', \vec{\omega})
\frac
{
	\Delta\Phi_{p}(x, \vec{\omega}_{p}')
}
{
\frac{3}{4} \pi r^{3}
}
\end{equation}

\subsection{Sub Surface Scattering}
Sub surface scattering is caused by a class of participating media where the light transport of the
material can be approximated by a function whereby the reflection of light at a point $x$ is
reflected at another point $x'$ This causes a softening of the apperance of the material a common
example of this is the human skin where a large amount of the reflected light by the skin undergoes
some amount of sub surface scattering. The reason that this case of participating media is
seperated from the general class is that their are specific properties of SSS that allow for
the calcualtion of the light transport more efficiently. Subsurface scattering can be described by
the bidirectional surface scattering function BDSSRF \cite{Jensen01}, this is a generalisation of 
the BDRF and is given as:

\begin{equation}
\label{eq:bdssrf}
	S(x_{i}, \vec{\omega}_{i};x_{o}, \vec{\omega}_{o}) =
		\frac
		{ dL_{o}(x_{o},\vec{\omega}_o)}
		{d\Phi_{i}(x_{i}, \vec{\omega_{i}})}
\end{equation}

From Equation~\ref{eq:bdssrf} it can be seen that their are two points that needed to form the
BDSSRF, the incidence location $x$ and the reflectance location $x'$, this can be seen as the
light incident to $x$ being scattered within the surface of the material and being reflected at
the point $x'$. The BDRF can be seen as a special case of this function where their is the
assumption that the location of emmited radiance is the same as that of incidence irradiance.
