\section{Photon mapping}
Photon mapping was first developed by Jensen \cite{Jensen95b} as an extension to the traditional ray
tracing algorithm proposed by Whitted \cite{whitted79a} whereby a preprocessing step is added that
creates a photon map of the scene.

The photon mapping algorithm begins by emitting photons from each light source in the scence,
each photon contains a fraction of the power of the light that is proportianal to the number
of photons that are emitted from the light source. Once emitted from the light source the
photon is traced through the scene much like a ray is traced from the eye, 
the photon is absorbed by a diffuse surface within the scene, each of the photons absorbed are stored within a photon map
which is an efficient data structure that will store information about the photon such as the
location of the absorbtion, the direction from which the photon was traveling and the photons
energy. the photon map is a spatial structure as we will query the photon map for the photons
within a location as such it is common to store the photon map in a kd-tree to allow an efficient
nearest neighbour search. Each photon emitted from the light source can be stored multiple times
along its path in this way we can use the photon map to estimate the radiance from all light
paths connecting the light and the eye.

Different types of photon maps are used with different accuracies in order to
improve the efficiency, for instance in \cite{Jensen96a} two photon maps are used, a global photon
map and a caustic photon map, the caustic photon map contains photons that arrive at a location from
either the specular reflections or refractions , this photon map is stored at a higher
accuraracy as its data will be visulaised directly to produce the apperance of caustics in an image.
Storing the caustic causing paths in a seperate photons not only makes the simulation of caustics
more efficient it also reduces the variance in renderings as the desity from all other paths
tend to vary more slowly \todo{cite}.

The general photon map contains all other types of photons stores, such as direct and indirect
illuminations and shadow photons as described in \cite{Jensen95c, Jensen96a} these shadow photons
allow for the number of shadow rays to be reduced as we do not need to shoot shadow rays if the
photon search returns only shadow photons.

The second stage in the photon mapping algorithm is the raytracing stage, this state is simalar
to the algorithm in \cite{whitted79a} but we now have the data stored within the photon map to make
use of. Rays are shot from the eye into the scene until an intersection is found, we then use
the propertiyes of object hit and the photon maps to calculate the colour set for the pixel, this
is done by estimating the radiance at the point of intersection, a nearest neighbor search is
performed on the photon map cetererd at the point of intersection, the radius of the search is
increased until the number of photons has reached a predetermined number $n$, the estimate of the
radiance is then given by the sum of the radiances of the photons within the search.

\begin{equation}
\label{eq:radiance_disk_estimate}
L_{r}(x, \vec{\omega})
\approx
\sum_{p=1}^n
f_{r}(x,\vec{\omega}_{p}', \vec{\omega})
\frac
{
	\Delta\Phi_{p}(x, \vec{\omega}_{p}')
}
{
\pi r^{2}
}
\end{equation}

The radiance is estimates as the density of a disk of radius $r$, this estimate is used as it is
assumed that the photons within the search will lie on the surface of the object and locally 
relitivly flat, this is accounted for by the division of $\pi r^2$ term in equation~\ref{eq:radiance_disk_estimate}.

The density estimate given in \cite{Jensen96a} can produce poor results in the case of geometry with
areas that are close together but not flat, such as the corners of a room, in this case the
estimate of the radiance will be too high as the photons lying on the other wall will contribute to
the estimate, a solution to this is given in \cite{JensenBook}, where the search radius is flattened along
an axis orthoganol to the normal of the surface. I am not sure if the computational cost of this
is worth the effort as more photons can be emmited which will increase the photon density thus
decreasing the search radius.
