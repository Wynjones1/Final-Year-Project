\section{Acceleration Structures}
In order to produce an image of a non-trivial scene accelerating structures are a nessesity, the
amount of time to check intersections for each ray grows with order  $O(N^2)$ if each object is
checked for an intersection. Some candidates for structures are, Octrees, KD Trees, BSP trees. All
of these data structures are designed to decrease the amount of time spent checking for 
intersections. It may be worth an investigation the performance characteristics of the various
data-structures within the system, this can easily be done by making the underlying implementation
of the data-structure transparent to the rest of the system.

\subsection{KD Trees}
A kd-tree is a data structure that partitions space along split-planes which are axis aligned
hyperplanes first preposed by Bently \cite{Bently75}.
The use of kd-trees in rendering applications is common as they reduce the time
required to find the intersection of a ray and an object within a kd-tree. kd-trees are also needed
within the photon mapping algorithm for the radiance estimate as we need to perform a
neighbour search, Jensen uses this data structure within his implementation of the photon mapping
algorithm in his book \cite{JensenBook}.

\subsection{Building KD-Trees}
Wald desribes an algorithm for building a kd-tree in $O(N log(N))$ \cite{Wald06}, as stated in his
paper the desire for more and more complex scenes requires that attention should be make to the 
building of the kd-tree as it can begin to take substantial amount of computing resources to
perform. This paper is a good overview of techniques for building a kd-tree.

\subsubsection{Surface Area Huristic}
The Surface Area Huristic (SAH) is a technique often used in the building of kd-trees in order to
decide upon the location of the split plane. First presented by MacDonald et al. \cite{MacDonald90}
the SAH estimates the cost of splitting at a given point on a kd-tree, this is done by estimating the
cost of traversing the structure and the probability of a ray intersecting the objects within the
structure. The SAH does make some assumptions such as the cost of traversel etc, as noted in
\cite{Wald06} some of these assumtions are not strictly correct but the result of using the SAH
are still valid.

\subsection{Nearest Neighbour search}
In order to produce a radiance estimate for a location on a surface we need to be able to query to
photon map in order to find the nearest n photons, the kd-tree is a good candidate for this search
as it is possible to perform the search in $O(log(N))$. The algorithm was first proposed in the
paper introducing kd-trees by Bently \cite{Bently75} although as noted by Bently in later versions of
the paper a more clear explination of the algorithm is stated by Freidman et al. \cite{Freidman77}.

\subsection{Ray KD-Tree Traversal}
Devoting time to creating efficient traversal algorithms is vital for all applications that
include raytracing as a large proportion of the running time will be used performing these
intersection tests, Fussell \cite{Fussell88} describes an algorithm to perform this traversal that
performs front to back search of the enclosing volumes allowing the closest intersection to be found
efficiently.
