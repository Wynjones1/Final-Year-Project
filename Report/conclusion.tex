The aim of this project was to produce a system that performs photon mapping on a scene description, this includes the simulation of
a variaty of lighting interactions in order to find an approximate solution to the rendering equation. We believe that the objectives
of the project have been met, 

\section{Assesment of Aims}

\section{Requirement Assessment}

\section{Devation from Plan}
Through the course of the project the scope it was nessaccary to evaluate the scope and direction of the project, the intiial plan
of the project was to include an alteration to the participating media phton mapping algorithm to increase the efficientcy of the
radiance estimate for certain light types (simulation of laser light) through the development of the project the complexity of
implementing a system with participating media and details of the system not considered during the inital planning stages and
investigating the suitability of the new approach in light of these details we decided to ommit the algorithm from the final product
of this project.

\section{Evaluation}
It is recognised that performing the
intersection test on each object is inefficient and that acceleration structures such as bounding volume heirachies would
have served to increase the performance of the system, we feel that due to the constriainst of the project that this
would give a smaller gain that conventraiting the effert of developme

\section{Future Work}
As with any project time constraints determin the number of features that can be reasonably be added, this includes algorithms that
would make the system more efficient such as irradiance caching

\section{Final Thoughts}
As a conclution to my degree studies this project has taught me a great deal and serves as a document of the progression in my technical and
non-technical skills, additionally the reasearch performed in preperation of the project and for the duration has showed my the
how areas of seemingly unrelated fields can come together to create some of the results described in this project, from monte carlo methods
first used during the Second World War as part of the allied efforts to construct the first neuclear weapon to \todo{Add something here}
