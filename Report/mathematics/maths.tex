In this chapter we will describe various mathematic concepts that are nessaccery in order to understand and implement photon mapping.

\section{Spherical Coordinates}
Spherical Coordinates describe a point with three values $r, \theta, \phi$, in this project we only consider coordinates with
radius equal to one, that is points on a unit sphere, this is due to using spherical coordinates for evaluating hemispherical
integrals, for all other coordinated standard carteisian coordinates will be used.

\missingfigure{Sperical Coordinates}

\section{Solid angle}

\section{Flux}
The photon mapping algorithm is based around emitting photons from light sources, the unit of measurement of a photon is
Radiant flux $\Phi$ the exact definition of flux is not relivant to the disscussion, but is essentially a measurement
of radiant energy with respect to time. \todo{cite}

\section{Radiance}
Radiance commonly referers to two quantities, incident radiance and exitance radiance, incident radiance being the radiance
that fall on a surface, exitance radiance being radiance from a surface (either from reflection or emission)

\section{Path Notation}
The radiance estimate on the right hand side of the rendering equation is typically evaluated by a recursive call to the
procedure that evaluates radiance at a point in the direction $\omega'$, as we perform more evaluation we create a path
from the eye to the final destination of the ray, it is frequently convienient to refer to these paths by the interaction
that the ray has with surfaces along the path, we use a regular expression language to express these paths, tokens in the
language include:
\begin{description}
\item[E] The eye or camera
\item[D] A diffuse surface
\item[S] A specular surface.
\item[L] A light
\end{description}

Paths begin with the eye and end at a light, a full solution to the rendering equation will evaluate the radiance due
to all paths \textbf{E(S\textbar D)*L}, raytracing evaluates all paths \textbf{ES*DL}

\section{Monte-Carlo Methods}
The rendering equation describes the light transfer as an integral of all directions above the point being evaluated, this
integral cannot be solved with a closed form solution, as a result we must approximate the value of the integral, to do this
we use monte-carlo methods. \todo{Add a citation}. Monte carlo methods allow us to estimate the value of an integral at a point
by sampling the 

\begin{equation}
E\left(g\left(X\right)\right) = \sum\limits_{x \in X} g\left(x\right)f_X\left(x\right)
\end{equation}


\subsection{Sampling}
\section{Importance Sampling}
Importance sampling is a statistical technique that allows us to concentrate computational effort when sampling a distribution
to areas of the distribution that contribute most to the average, for instance sampling incident radiance at a point on a lambertian
surface where the reflected radiance is preportional to the cosine of the incoming direction, by using a cosine weighted distribution
we can gain a better estimate for the radiance.

\todo{Motivations}
\section{Rays}

\section{Intersection Tests}
In this project we represent the objects in the scene with implicit and explicity definition of surfaces, this includes
triangle meshes and spheres, each of these objects in order to be used in the raytracing system a function that tests
whether a given object in the scene intersects with a given ray.
\subsection{Triangle}
The intersection test used for triangle intersection tests in this project was derived by \todo{Cite}
\subsection{Sphere}
The intersection of a ray and a sphere is one of the simplest, it is essentially finding the root of a quadratic equation to
find the ray parameter at the point of intersection, if no value can be foudn the ray does not intersect the sphere.
\subsection{K-D Tree}
In order to accelerate the intersection of meshes that contain many traingles we use a k-d tree to spacially divide the mesh,
this structure facilitates an efficient intersection tests that calculates the path that the ray travels in order to perform
a front to back tests, this allows for an early exit of the intersection test when an intersection is found in a partition.
The algorithm is as follows, we begin at the root of the k-d tree, testing which side of the spliting plane the ray origin
is situated, we set this node as the near node, this will be the node that will be tested first. We calculate the ray parameter

\todo{Expand}

