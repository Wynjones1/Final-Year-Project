In this chapter we will describe various mathematics concepts that are necessary in order to understand and implement photon mapping.

\section{Spherical Coordinates}
Spherical Coordinates describe a point with three values $r, \theta, \phi$, in this project we only consider coordinates with
radius equal to one, that is points on a unit sphere, this is due to using spherical coordinates for evaluating hemispherical
integrals, for all other coordinated standard Cartesian coordinates will be used.

\missingfigure{Spherical Coordinates}

\section{Solid angle}
\todo{Picture and explain why we use it and some nice citations from someone other than Jensen}



\todo{Move}
\section{Monte-Carlo Methods}

%\section{Intersection Tests}
%In this project we represent the objects in the scene with implicit and explicity definition of surfaces, this includes
%triangle meshes and spheres, each of these objects in order to be used in the raytracing system a function that tests
%whether a given object in the scene intersects with a given ray.
%\subsection{Triangle}
%The intersection test used for triangle intersection tests in this project was derived by \todo{Cite}
%\subsection{Sphere}
%The intersection of a ray and a sphere is one of the simplest, it is essentially finding the root of a quadratic equation to
%find the ray parameter at the point of intersection, if no value can be foudn the ray does not intersect the sphere.
%\subsection{K-D Tree}
%In order to accelerate the intersection of meshes that contain many traingles we use a k-d tree to spacially divide the mesh,
%this structure facilitates an efficient intersection tests that calculates the path that the ray travels in order to perform
%a front to back tests, this allows for an early exit of the intersection test when an intersection is found in a partition.
%The algorithm is as follows, we begin at the root of the k-d tree, testing which side of the spliting plane the ray origin
%is situated, we set this node as the near node, this will be the node that will be tested first. We calculate the ray parameter
%
%\todo{Expand}

