\section{Spherical Coordinates}
\section{Solid Angle}
\section{Radiance}
Radiance commonly referers to two quantities, incident radiance and exitance radiance, incident radiance being the radiance
that fall on a surface, exitance radiance being radiance from a surface (either from reflection or emission)

\section{BRDF}
\section{The Rendering Equation}
\section{Lambertian Surfaces}
\section{Monte-Carlo Methods}
The rendering equation describes the light transfer as an integral of all directions above the point being evaluated, this
integral cannot be solved with a closed form solution, as a result we must approximate the value of the integral, to do this
we use monte-carlo methods. \todo{Add a citation}

\section{Importance Sampling}
When sampling in order to evaluate integrals such as in the previous subsection we can reduce the variance in the estimate by
sampling based on the importance of the distribution such that directions that 

\todo{Motivations}
\section{Reflection and Refraction}
\section{Intersection Tests}
\subsection{Triangle}
\subsection{Sphere}
\subsection{K-D Tree}
In order to accelerate the intersection of meshes that contain many traingles we use a k-d tree to spacially divide the mesh,
this structure facilitates an efficient intersection tests that calculates the path that the ray travels in order to perform
a front to back tests, this allows for an early exit of the intersection test when an intersection is found in a partition,

\todo{Expand}
