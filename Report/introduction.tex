Computer graphics is a sub field of Computer Science that has obvious appeal, that of creating interesting images with computers. Throughout
this century and the latter half of the last researchers have strived to develop algorithms that simulate how the real world works
and use these algorithms to produce images that are visually appealing and that appear to be in some way realistic.

\section{Global Illumination}
As the use of computers has become more prevalent in everyday life so has its use in the arts, a leading area of research is
computer generated imagery we the aim of producing images that simulate the way in which light interacts
with the world, global illumination techniques aim to achieve this through evaluation of the light transport equation of a scene. A true
solution to this equation is computationally unfeasible as a result techniques have been developed that estimate the contribution
due to indirect lighting that seek to reduce the computational cost that give a correct solution to the rendering equation at
the limit, one such technique is photon mapping, this is the focus of this project.

\section{Aims}
This project aims to create a system that will synthesis images using the photon mapping algorithm to estimate the
global illumination of a scene, in particular we aim to simulate lighting phenomena such as colour exchange between
diffuse surfaces, caustics caused by secular objects. We aim to create a system that is by some measure efficient
and that is easy to use.

\section{Structure}
This document will begin with a description of the research surrounding my project that we have used to draw from. We will then
move onto the requirements for the system then a description of the design and then the implementation of the system including
aspects of the implementation that were interesting or presented a particular challenge. Finally we shall provide a brief description
of the testing performed on the system.
