\newcommand{\req}[2]
{
	\item {\it #1} \hfill \\ #2
}
  
\subsection{Functional Requirements}

\begin{enumerate}[{1}.1]
	\req{Must Support Image Output.}
	{
		The program is designed to synthesis images, as such it is important to be able to
		save the images produced by the program.
	}

	\req{Must Support GUI output.}
	{
		It is desirable when creating an image to be able to see the results of the render as they are created.
	}
	\req{Scene input must be in human-readable format}
	{
		This is so as editing scene files is possible without the use of other programs.
	}

	\req{Must accept all scenes in valid format}
	{
	}

	\req{Must not accept invalid scene files}
	{
		The user should be informed of invalid input.
	}

	\req{Must perform photon mapping algorithm}
	{
		As this is the focus of the project it is a key requirment of the final product.
	}

	\req{Must support diffuse and specular reflectance properties}

	\req{Must support generic scene objects}
\end{enumerate}

\subsection{Non Functional Requirements}

\begin{enumerate}[{2}.1]
	\req{Must Support Linux and Windows}
	{
		Cross platform operation of the program must be possible.
	}
	\req{Should Support Mac OSX}
	{
		Due to hardware availability this is less strong of a requirement than 2.1
	}
	\req{Should support user configuration from the command line}
	{
		Users should be able to configure the render parameters without recompiling.
	}
	\req{Should perform synthesis in a reasonable time}
	{
		The use of algorithms and acceleration structures that lower rendering times
		should be used where possible.
	}
	\req{GUI should be responsive to user interaction}{}
	\req{Results must be consisten across runs}
	{
		The program should not demonstrate different behaviour for different runs with
		the same input data.
	}
	\req{Scene input format should be documented.}
	{
		Users should not be expected to read the source code in order to create
		a scene for the program.
	}
	\req{Software design must be modular with clear seperation of concerns}
\end{enumerate}
