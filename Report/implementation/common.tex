In this section I will briefly discuss some data structures that are used as part of the system that cannot
be categorised as being in the front or the back end.

Due to the small standard library provided by C many features that are commonly found in more high level languages are not
found in C, for this project we need three such data structures, lists queues and vectors.

\subsection{Lists}
As we don't know how many objects and lights are in the scene and number of photons stored in the photon maps depends on the
lights and geometry in the scene, as a result we have implemented a list data structure that allows for the size of the list
to increase as elements are added to the list. The implementation of this data structure is quite simple, a list stores
a pointer to an array in memory, the size of the elements in the array, the number of elements in the array and the capacity of the
array. If an item is added to the list and the capacity is two low the capacity is increaed by calling the function \texttt{realloc}
allowing for dynamic memory managment to be encapsulated.

\subsection{Queue}
During the implementation of the system we frequently required to coordinate between multiple threads for example during
photon emission and ray-tracing, the solution that we decided upon in both cases was to use a thread-safe queue implementation, this is implemented as a
fixed size circular buffer, two pointers are stored to the front and back of the queue, as items are read from the queue to
front of the queue is decreased and writing to the queue increases the back pointer.

\subsection{Vectors}
Some of the most common operations that are performed during photon mapping and ray-tracing are vector operations. We provide a simple
implementation of vectors in this system as facilitate operations such as vector addition, subtraction, dot and cross product and so on. The
implementation of these operations are not particularly optimised although we were able to observe that our implementation was optimised
by the compiler to utilise vectorisation instructions available for the architecture that we have tested the system on.
