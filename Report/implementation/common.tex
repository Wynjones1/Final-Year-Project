In this section I will briefly discuss some data structures that are used as part of the system that cannot
be catagorised as being in the front or the back end.

Due to the small standard library provided by C many features that are commonly found in more high level languages are not
found in C, for this project we need three such data structures, lists queues and vectors.

\subsection{Lists}
As we don't know how many objects and lights are in the scene and number of photons stored in the photon maps depends on the
lights and geometry in the scene, as a result we have implemented a list data structure that allows for the size of the list
to increase as elements are added to the list.

\subsection{Queue}
During the photon mapping stage of the back end we need to coordinate a number of threads that is dependant on runtime inputs,
the problems that we faced during the development were distribution of work and coolation of the results of each thread,
the solution that we decided upon in both cases was to use a thread-safe queue implementation, this is implemented as a
fixed size circular buffer, two pointers are stored to the front and back of the queue, as items are read from the queue to
front of the queue is decreased and writing to the queue increases the back pointer.

\subsection{Vectors}
Some of the most common operations that are performed during photon mapping and raytracing are vector operations.
