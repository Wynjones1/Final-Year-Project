\section{Tools} 

\subsection{Implementation Language (0.5 pages)}
The implementation language that I have chosen is C, specifically C11. The choice of C is because it it the
language that I am most familiar with, also it is a language that offers performance benifits as it is a compiled
language, this is of great advantage for a photon mapper as we will regularly be dealing with data structures
that contain thousands to millions of elements (such as the photon map.) In addition to being a compiled language
the memory managment of the program is left to the programmer, while this can cause issues such as buffer overflows
it has the benifit of providing a great deal of control to the program writer as they can be aware of all memory
that is used by the program, which, when dealing with data structures as large as the photon map is a large benifit.

\subsection{Coding Style and Code Quality}
\todo{This will need some citations at some point}
With any piece of software it is important to impose restrictions on the way in which the code is written in order
to reduce the effort that is needed to read the source. Some of conventions that I have used includes having a
convention for any typedef such that a type can be easily deternimed (appeding \_t to the end of the typename),
requiring that if a function is non-static that it begin with the name of the file or another prefix to prevent
name clashing. Another convention that is used is the use of opaque pointers to data structures in order to
hide the internal data representation of the types an example of this can be found in the source files \texttt{list.h/c}
it can be seen that the header file does not expose the structure definition to any file that includes the header.

Throughout the development of the system it was nessacarry to use a number of tools in order to ensure the
quality of the code being produced.

Another issue that is present in writting code for C is that the memory is managed by the programmer, as a
result it is common for errors such as memory leaks to occur, in order to reduce the risk of this in the project
I have utilised several programs that can identify issues.

\subsection{Compilers}

The first tool that I used was the compiler itself, there are several options available that will detect and report
common error in C programs such as uninitialised variables.

\subsection{Valgrind}
In order to detect memory related bugs I used Valgrind which is a program that will track all memory allocations
of a running program and is able to detect errors such as writes after a region of memory that has been allocated
and lost pointers that result in memory leaks.

\begin{figure}
\lstinputlisting[language=C]{./implementation/val.log}
\label{fig:valgrind_listing}
\caption{Example Valgrind Output}
\end{figure}

\subsection{Static Analysis}
Included with the clang compiler suite is the clang static analyser, this program will perform analysis on the
source code of a program and will report certain types of errors such as performing memory accesses on variables
that can potentially be NULL, while this may seem simalar to valgrind, this approach will detect errors such
as this without running the program and can find issues that may only manigest in certain corder cases. Below
is an example of a fix that was made that was the result of a warning from the static analyser that proved
useful as the issue that was flaged and at first assumed to be impossible occured, this issue had the potential
to be quite hard to find.

\subsection{Source Control}
In order to orginise the code that I have created I decided to use Git as a source control system for the project.
Git has the concept of branches that allow development of seperate features concurrently in isolation and
merged upon completion, incorperating branched into the development workflow proved to be a highly useful feature
as it allows for work on a feature to continue even if another feature breaks parts of the system that would
otherwise make continurd development impossible

\subsection{Profiling}
When writing software it is desirable to optimise it in order to improve the preformance of the code, this project
is no different and there are many area that could be optimised, to quote Knuth

\begin{quotation}
{Premature optimization is the root of all evil ... in programming \cite{Knuth74a}}
\end{quotation}

This refers to the practice of optimising code that is not a bottleneck in the execution of the program, for instance
reducing the running time of a function by $50\%$ that is only executed for a small preportion of the running time of
the whole program in order to decide which parts of the code needs to be optimised we have used the combination of
the compiler profiling flag and gprof, a program that will output profiling statistics for a run of the executable,
this includes information such as the number of times a given function was called and the percentage of the run time
taken by the function.

\newpage
\section{Common Data Structures}
In this section I will briefly discuss some data structures that are used as part of the system that cannot
be catagorised as being in the front or the back end.

Due to the small standard library provided by C many features that are commonly found in more high level languages are not
found in C, for this project we need three such data structures, lists queues and vectors.

\subsection{Lists}
As we don't know how many objects and lights are in the scene and number of photons stored in the photon maps depends on the
lights and geometry in the scene, as a result we have implemented a list data structure that allows for the size of the list
to increase as elements are added to the list.

\subsection{Queue}
During the implementation of the system we frequently required to coordinate between multiple threads for example during
photon emission and raytracing, the solution that we decided upon in both cases was to use a thread-safe queue implementation, this is implemented as a
fixed size circular buffer, two pointers are stored to the front and back of the queue, as items are read from the queue to
front of the queue is decreased and writing to the queue increases the back pointer.

\subsection{Vectors}
Some of the most common operations that are performed during photon mapping and raytracing are vector operations.

\newpage
\section{Front End (3 pages)}
This section described the implementation aspects of the front end of the system. As the front end is primaraly responsible
with the i/o of the system it is unlikely to be a bottleneck in the performance of the system, as a result little effort
has been expended into optimising the front-end.

\subsection{Scene Input (1.5 pages)}
The scene input is read into a data structure that is passed to the back-end, this structure is implemented as a C struct
containing a list of all objects in the scene, a camera definition that defines how the eye rays are created.

Mesh input to the scene allows for each of the meshes to have matrix operations applied to them, these trasformations
are applied at the construction of the mesh prior to creating the kdtree of the mesh.

\subsection{Command Line (0.1 pages)}
Users are able to input command line arguments to the system, these are received by C programs in the input parameters of
the main function, these input parameters need to be parsed and stored, this is performed by simply scanning over each
option in the argv list to find a valid option or pair of options in the case of a user inputed value for the option
i.e. width of the output image.

\begin{figure}
\texttt{./raytracer -w 1000 -h 1000 -i ./data/scenes/cornell\_box.scene}
\caption{Example command line.}
\end{figure}

\subsection{Global Configuration}
Once the configuration of the scene has been read and processed the data containing the configuration will not change for
the lifetime of the program, as a result the global configuration is available in the system through a global variable, while
it is generally advised whene developing software that global variable should be avoided an implementation that passed the
configuration would require that a pointer to a configuration structure be passed, this amounts to having the same result
as a global variable whilst reducing the clarity of the code.

\begin{figure}
\centering
\includegraphics[width=\textwidth]{./images/pixel_update_sequence.png}
\caption{Sequence for pixel update}
\end{figure}

\subsection{Common Pixel Update Interface}
In order to make integration of the system as simple as possible we have created an interface that will encapsulate the data
that is passed from the backend to the front-end to be presentend to the user, this facilitated the decoupling of the front
and back-end. Two modules have been created that use the interface to demonstrate the useage of the interface, these are
desctibed below. The common pixel output is implemented by registering output from the system through a function pointer,
this pointer will be called whenever a pixel is received from the back-end.

\subsubsection{Image Output (0.5 pages)}

The systems output image type is BMP (extension .bmp) this image format was chosen as it is a simple image format, there are
several variants of BMP each with its own header definition the version of BMP that is implemented is the OS/2 bitmap this is
one of the more simple variants with a fixed size header that specified few options. The implementation of the BMP output can
be found in \texttt{bmp.h/c}. If supported on the platform the system is also capable of outputing images in PNG file format,
this is acheived by using libPNG, there are advantages to outputing to this format, foremost is that the image size of an
image output to PNG is significantally smaller than that outputed by BMP.

\todo{maybe include size graph (might not be relivant though)}

\subsubsection{GUI (0.5 pages)}

The User Interface That is included in the system is designed to allow instant feedback to the user of the render, this can be
useful in the case of errors that can be identified early in what may be a costly render, the interface is rather simple
providing only a surface that displays the current state of the render.

I have decided to use SDL library to provide the interface to the operating system window system and OpenGL to perform the 
drawing of the pixels to the screen.

SDL (Simple Direct media Library) is a library that allows for cross platform applications to be written that include
drawable surfaces, OpenGL support and user input, SDL is written in C and provides a library API for C.

In order to retain responsivness of the GUI while rendering it is desirable to run the GUI on a seperate thread, the GUI
should also only redraw the screen when a pixel has updated its colour, this is to reduce the amount of processing that
the GUI is performing that could be used to create images, on the other had the GUI should respond to events that are pushed
to its internal event queue such as key presses and window system events.


\newpage
\section{Object System}
In order to meet requirement \todo(find requirement to ref) the system needs to be able support arbitrary objects descriptions
that can be traced in the scene, this is implementatied as a structure that contains function pointers that perform functions
that must be supported by all raytracable objects and the data related to the individual object.
By encapsulating the data and the functions performed by the object it is possible to create algorithms on objects in the
abstract such that the details of the object can be ignored, for instance calculating a reflected ray at a point if intersection
is a function of only the surface normal at that point, wheither it is a triangle mesh or sphere the same ray will be calculated,
a full list of the functions that need to be defines is given in Figure~\ref{fig:object_funcs}

\begin{figure}[h]
\begin{description}
\item[int  (*intersection\_func)(object\_t *o, ray\_t *ray, intersection\_t *info)] \hfill \\
	Tests if the input ray intersectes with the object, returns the result and any other information is stored in the input variable info
\item[void (*bounds\_func)(object\_t *o, aabb\_t *bounds)] \hfill \\
	Returns the bounds of the object which is used during the scene-ray intersection test.
\item[void (*normal\_func)(object\_t *o, intersection\_t *info, double *normal)] \hfill \\
	Returns the normal of an object at a given intersection point defined in info.
\item[void (*tex\_func)(object\_t *o, intersection\_t *info, double *tex)] \hfill \\
	Returns 2-d texture coordinates in tex of the object at the point of intersection info.
\item[void (*delete\_func)(object\_t *o)] \hfill \\
	Frees any memory and other resources used by the object.
\item[void (*shade\_func)( object\_t *object, scene\_t *scene, intersection\_t *info)] \hfill \\
	Returns the colour at the surface of the object at the intersection point defined in info the result is stored in info.
\end{description}
\label{fig:object_funcs}
\caption{Functional Definition of an Object}
\end{figure}


We will now attempt to give an overview of the implementation of the objects currently used within the system.

\subsection{Sphere}
The sphere primitive is defined implicitly as an origin point and distance from the origin, including a sphere primitive is
a natural choice as many of the operations that we need to perform such as intersection tests and finding the normal at the
point of intersection is much simpler for spheres that most other surfaces allowing us to use this primitive for testing
changes while being confident that any errors that we find are not within the object definition of the sphere, the same cannot
be said for a mesh where the optimised intersection test used is fairly complex and contains more scope for implementational errors.

\subsubsection{Intersection test}
Given that we define a ray and a sphere in a parametric form it is intuitive to solve the intersection of these objects implicity,
given a ray defined $\vec{o} + t \vec{d}$ and the equation for a sphere $x ^ 2 + y ^ 2 + z ^ 2 - r ^ 2 = 0$

\subsection{Mesh}
The mesh primitive in the system is defined as a collection of vertices connected into triangles through a list of triangles,
additionally surface normals and texture coordinates can be specified in the input file (see section \todo{find section})

\subsubsection{K-D Tree Construction}
In order to satisy requirement TODO:Add the requirement bruteforce methods for raytracing are not acceptable, in order to reduce
to number of intersection tests that are performed I have decided to use the kdtree for the accelleration structure. The
construction of the kdtree is performed during the initialiseation of the triangle meshes as it is a static data structure, due
to it being a static data structure their is an advantage to performing more extensive computation in the construction in
order to create a higher quality of kdtree. TODO:Add sah and analysis. We begin the contruction of the kd-tree with all
triangles in a single list, we then calculate an axis and position on that axis to seperate the triangles, this is calculated
by \todo{add SAH or average if not implemented} we then sort each of the triangles into two child lists, if a triangle is
to the left of the splitting plane it will be placed in the left childs triagnle list and visa-versa for the right, if the
triangle spans the split plane it will be added to both lists, we then call the tree building function recursivly on the
left and right child and delete the list for this node, we terminate when a list contains less than a threshold number
of triangles in its bucket or a maximum depth is reached, or if all triangles are placed in a single node.

\subsubsection{Intersection test}
Given the k-d tree that we have constructed for a mesh we can now describe an efficient top-down intersection algorithm,
we begin at the root of the k-d tree, given that the tree has been split along an axis we can calculate which of the
child voxels are closer to the ray by calculating the values of intersection for the two voxel, if the ray origin is
within a voxel it is set to be the near voxel and the other child the far, otherwise we compare the t values for the
voxel intersection. Once we know which voxel to traverse, if the ray exits the near voxel before crossing the splitting
plane we do not need to test the far voxel as we cannot intersect any geometry in that voxel, we perform this procedure
recursibly on the near and possibly far voxel until we reach a leaf node, we then test all triangles in the index list,
if we find an intersection we record the t parameter for the intersection and check the rest of the list, if an intersection
if found we can return the intersection point and discard any other voxels that we were going to check as they cannot
have an intersection closer. There is a subtle point that must be noted, when we are checking for that intersection of
a triangle in a list if the intersection point is found to be more that the exit point of the voxel then we do not
record this intersection as the triangle spans voxels and we need to check for closer intersection.


\subsection{Material Properties}
Each object has associated with it a material property that defines the reflectance properties of the material.
The system currenlty suports diffuse and pure specular components each being defined as a three floating point
values for red, green and blue compnonets respectivly, texture mapping is also supported for the diffuse component,
for specular transmission an index of refraction is defined that determins the direction of refracted rays through the
material. We also store the average values for these components explicitly as they are frequentlty used when performing
russian roulette to evaluate integrals during photon map construction and raytracing as will be seen in section \todo{Add section}

\subsubsection{Participating Media}
Materials can also be set to being a participating media, in this case we have four coefficitnes, the scattering and
absorption coefficient which will determin how light interacts with the object internally derived from these values
are the extinction coefficient and albedo we will demonstrate their useage in section \todo{section}.



\newpage
\section{Back End 17 pages}
As the back end is responsible for performing the image synthesis it contains the majority of the code that requires disscussion.
\section{Photon Map Generation}
We described in Chapter~\ref{chap:design} that the photon generation stage of the back end would run on multiple threads
to utilise the computing power of the machine the system is being run on, this required certain descitions to be made to
allow for the multiple threads to correclty distribute the power of the light sources into the scene.

Each thread is responsible for producing a certain proportion of the photons in the scene, once a thread has processed all
of these photons it will signal to the main thread that it has finished by sending a flag to the output queue the thread also
updates a global light emission count include the photons that the thread emitted for each of the lights, a seperate count for
each of the counts is kept as the processing of the threads may not happen at the same time.

\subsection{Photon Emission}
In order to trace photons into the scene we first need a method of creating photons, this is implemented in the light 
defintion, each light object has a associated function that will produce a random photon from the light that is consitent with 
the type of light, for instance a point light will generate a photon with origin exactlu at the origin of the light source and
in a random direction, an area light produces a photon with origin on the area of the light with direction taken from a
cosine weighted hemisphere distribution in the direction of the normal of the light. Each photon that is emitted from the
light sources begin with the full power of the light source, this will be scaled by the number of emitted photons after
all photons have been gathered.

\subsection{Photon Tracing}
Once we have created a photon from a light source we now need to trace the photon through the scene and record the interactions
of the photon until it is absorbed or leaves the scene, the function that performes this function is \texttt{trace\_photon} the
declaration of this function is given below.

\texttt{int trace\_photon(scene\_t *scene, ray\_t *ray, int light, double power[3], bool specular, bool diffuse, bool specular\_only};

The parameters ray, light and power defines the photon properties, ray defines the origin and direction of the photon, light is an index
to the light from which this light was emited and the power contains the flux of the photon. specular and diffuse
define describe the path that the photons have taken prior to the call to \texttt{trace\_photon}

Photons are traced much like a ray in traditional raytracing, first we calculate the closest intersection point of the photon
once found the photon interacts with the surface of the object at the point of intersection this interaction can be specular 
reflection , transmission, diffuse reflection or absorbtion, at this point of intersection we must determine the path of the reflected
photon, the power of this photon is determined by the reflectance of the surface i.e a photon with white light that interacts with
a red surface will only reflect photons with power in the red component. Using the approach of scaling the photons will create a photon
map that correctly describes the distribution of power in the scene, unfortunatally this can lead to many photons with low power as they
are scaled at each interaction, it can also in the case of surfaces with both specualr and diffuse componenet lead to exponenetial number
of photons being produced as we need to create a reflected photon for both paths scaled apporoaprealty, as a result of these considerations
we have used a monte-carlo method that reflectes at most one photon
per surface interaction with full power, we ensure the correct result in the photon map by only reflecting the photon
with probability based on the reftance of the surface (i.e reflecting 50 full power photons as oppose to 100 full power photons)
we sample the reflectance of the surface by forming a cumulative distribution funciton of the materials reflectance coefficients.
Given the specular reflecive, transmissive and diffuse reflectance coefficents $\rho_{s}, \rho_{t}, \rho_{d}$
of the object we can create a cumulative distribution of photon emittion paths (Figure~\ref{fig:rr_dist}) 
we then take a uniform variable $\xi$ between 0 and 1 if the variable is within the distribution we reflect a photon, otherwise
the path of the photon is terminated. As we are using rgb values to store reflectance properties the
coeffients used to perform the sampling are the average of the three componenets, so as the distribution of the three
colour bands is correct we must also perform scaling of the power parameter that we use in the next \texttt{trace\_photon}
call, note that the scaling preserves the power of the photon and the reflectance properties of the surface.

\begin{figure}[h]
\includegraphics{./images/russian_roulette_distribution.png}
\caption{Russian roulette distribution}
\label{fig:rr_dist}
\end{figure}

At each non-specular interaction the photons are written to the output queue to be processed, this includes
the power of the photon, the incident angle of the photon to the surface and flags to indicate if the surface
has interacted with a diffuse or specular surface prior to this interaction, we do not store photon interactions
at specular surfaces as they do not provide us with information that can be used when estimating the radiance as
the probability of an incoming photon contributing to the specular reflectance is zero due to the delta functions
in the specular BRDF.

When creating the caustic photon map we are only concerned with those interactions with diffuse surfaces that have
taken a path that contains at least one specular bounce and no diffuse bounces (\textbf{LS$^+$DE}), as a result \texttt{trace\_photon}
has as input a flag that will indicate that any path that does not match this description should be discarded.

\subsubsection{Participating Media}
If the photon interacts with a participating medium the photon will begin a random walk inside the medium, the path of this walk
is determined by the properties of the medium, these being the scattering and absorbtion coefficient, the sum of which
is called the extinction coefficient, at each point in the random walk the photon is stored and then can either be
scattered or absorbed, the probability of being scattered is determined by the Albedo $\Lambda$

\begin{figure}
\centering
\includegraphics[width=0.5\textwidth]{./images/random_walk.png}
\caption{Random walk of photons through participating media}
\label{fig:random_walk}
\end{figure}

\begin{equation}
\Lambda = \frac{\sigma_s}{\sigma_e}
%\caption{Participating Media Albedo}
\end{equation}

Each step in the random walk continues for a distance until the next interaction with the medium is calculated, this distance
is calculated by importance samping the distance $\Delta x$ such that the expected distance is equal to that of the medium \cite{JensenBook},
the distance is given by:

\begin{equation}
\label{eq:volume_dist_importance}
\Delta x = \frac{-log(\xi)}{\sigma_t}
\end{equation}

where $\xi$ is a uniform random variable in the range [0, 1]. If a photon leaves a participating medium or intersects with an
object in the medium it is treated as if it was not in the medium and is stored in the global photon map.

\subsection{Photon Processing}
Each thread that traces the photons in the scene writes to a common queue, this queue is read by a single processing thread
that clasifies the photons by the path of the photon and stores the photons in one of three lists, the global, caustic and
volume photon list, this thread will continue to read photons from the queue until each of the threads have signaled that they
have completed their portion of the photons, this signal is in the form of a flag that is passed in the same queue as the
photon data. After all photons have been collected the power of the photons are scaled by the number of photons that were
emmited (note this is not the number of photons in the list but the total number emmited) from the same light as the
photon.

\subsubsection{K-D Tree Balancing}
When performing radiance estimations with the photon map we will be performing nearest neighbour searches on points within
the map, this requires the photon map to be arrainged in a manner such that this search can be performed efficiently.
In order to do this we store the photon map in a left-balanced k-d tree. A left balanced tree is a tree structure where
at each level of the tree the depth of the children differs by at most one,
this allows us to store the photon map in an array with the location of the children in the photon map known implicitly,
for a photon in position $i$ the children of the photon can be found at the $(2i + 1)^{th}$ and $(2i + 2)^{th}$
location for the left and right tree respectivly, the next stage is two transform the photon lists into a k-d tree.

\begin{algorithm}
\begin{algorithmic}
\caption{Balanced K-D tree construction}
\label{alg:balance}
\label{alg:balance}
\Function{balance}{photons, index, max\_index}
\If
{
photons.size \textgreater 1
}
{
	\State axis $\gets$ \Call{select\_axis}{photons}
	\State left, median, right $\gets$ \Call{partition\_around\_median}{photons, axis}
	\State left\_index  $\gets$ 2 * index + 1
	\State right\_index $\gets$ 2 * index + 2

	\If{left\_index \textless max\_index}
		\State left $\gets$ \Call{balance}{left, left\_index,max\_index}
	\EndIf

	\If{right\_index \textless max\_index}
		\State right $\gets$ \Call{balance}{right, right\_index, max\_index}
	\EndIf

	\Return{left + median + right}
}
\Else
{

	\Return{photons}
}
\EndIf
\EndFunction

\end{algorithmic}
\end{algorithm}

\begin{figure}
\centering
\begin{subfigure}{0.4\textwidth}
\includegraphics[scale=0.8]{./images/left-balanced-tree.png}
\caption{}
\end{subfigure}
\begin{subfigure}{0.4\textwidth}
\includegraphics[scale=0.8]{./images/non-left-balanced-tree.png}
\caption{}
\end{subfigure}
\caption{Left Balanced Tree}{(a) demonstrates a left balanced tree while (b) does not due to node F}
\end{figure}

\subsubsection{Selection Statistic}
In Algorithm~\ref{alg:balance} we partition the photons around the median of the list of photons, in order for
this partition to be stored in an array with no explicit node pointers we must choose the median position such
that the tree will be left balanced \cite{baerentzen03}, Algorithm~\ref{alg:bal-median} describes the procedure nessacerry for this
to be the case.

\begin{algorithm}
\caption{Balanced Median Calculation}
\label{alg:bal-median}
\begin{algorithmic}
\State $n \gets 0$
\While{$2^{n + 1} <= N$}
	\State n $\gets$ n + 1
\EndWhile
\State $M \gets 2^n$
\State $R \gets N - (M - 1)$
\If{$R <= M / 2$}
	\State \Return $(M - 2) / 2 + R$
\Else
	\State \Return $(M - 2) / 2 + M / 2$
\EndIf
\end{algorithmic}
\end{algorithm}



\section{Raytracing (10 pages)}
\todo{Add the algorithm back in}
%\begin{algorithm}[H]
%\For{each pixel in output image}
%{
%	send pixel x, y to thread queue;
%}
%
%finish\_count := 0;
%
%\While{finish\_count != thread\_count}
%{
%	data := read from output queue
%
%	\If{data == pixel\_data}
%	{
%		send pixel data back to front end
%	}
%	\Else
%	{
%		finish\_count := finish\_count + 1;
%	}
%}
%
%\caption{Raytracing Algorithm}
%\end{algorithm}

\subsection{Algorithmic Overview}

\subsection{Threading Model (0.5 pages)}
As noted in the design section of this document raytracing related rendering methods are inherently easily parrelised as each
ray can be traced in its own thread, the method that the system uses to distribute the work that is needed to be done is
based on a producer consumer model whereby the work is fed into a queue of work items that can be read by any of the worker
threads, in order to perform this a queue implementation was needed, this can be found in the \texttt{queue.c} source file,
this queue is thread-safe and as such is suitable for this use case. When working with multiple threads it is important
to prevent as much global state as possible, within the system this is no different and as a rule global variables are not used
with the exception of the configuration structure, this is due to the fact that after the initial setup of the scene the data
during the start of the programs execution no data within the structure is permitted to be modified and as such can be read
safety from any thread. In order to inform each thread that all pixels have been processed a dummy input is sent to each of the
threads, on reading this input the thread will return.

\missingfigure{Threading Model}

The choice of mulitthreading library that I have used is pthreads. Posix threads are a definition
of a threading model that allows C programs to be run on muliple threads, pthreads are defined for unix like systems so
linux and mac os are supported so partially fulfills requirement \todo{add requerement refs}: as windows does not have native
support for pthreads, there is however an implementation that uses native Windows threads to present the types and functions defined
by pthreads. As I am using C11 it could be possible to use the threading functionallity that is defined in the standard,
unfortunatally the support for this feature is largly unsupported (as of \today).

\subsection{Pixel Sampling}
As the raytracer is performing distribution raytracing each thread will perform multiple raytracing operations for each pixel
that it processes each sample per pixel is offset inside the pixel to perform antialiasing, this reduces the sharp edges that
can be seen in images. In order to improve the appearence of the image we also perform jittering on the samples this reduces
certain artifacts such as banding. \todo{Make sure the banding comment is correct and if so find a reference for it}

\missingfigure{anti-aliasing}

\subsection{Intersection Storage}
When raytracing the scene the intersection point of the ray needs to be calculated and stored, this will include data that
is specific to an objects intersection, for example an intersection with a triangle mesh will need to record the trangle
that was intersected and as a result will need to record the barycentric coordinate at the point of intersection.

\subsection{Object System}
In order to meet requirement \todo(find requirement to ref) the system needs to be able support arbitrary objects descriptions
that can be traced in the scene, this is implementatied as a structure that contains function pointers that perform functions
that must be supported by all raytracable objects and the data related to the individual object.

\begin{description}
\item[normal] returns the normal at a point of intersection.
\item[bounds] return the bounds for the object.
\item[texture] return the texture coordinates at a point of intersection
\item[intersection] returns if a ray intersects a given object.
\end{description}

By encapsulating the data and the functions performed by the object it is possible to create algorithms on objects in the
abstract such that the details of the object can be ignored, for instance calculating a reflected ray at a point if intersection
is a function of only the surface normal at that point, wheither it is a triangle mesh or sphere the same ray will be calculated.

\subsubsection{Texture Mapping}
Performing texture mapping requires the ability to query the texture coordinates at a point of intersection, for mesh objects this
will interpolate the barycentric coordinates for the triangle of intersection, spheres use a spherical mapping that uses the
spherical coordinates to generate u,v values.

\subsubsection{Mesh}
The mesh type is defnined as a list of vertices and triangles constisting of indices into the list of vertices.

\subsubsection{Sphere}
The sphere type is simply defined as a origin point and radius.

\subsubsection{Participating Media}
The participating media type is created by applying a participating media materail property to an object, this allows
for arbitrary geometry to be used as a participating media.

\subsection{K-D Tree Construction}
In order to satisy requirement TODO:Add the requirement bruteforce methods for raytracing are not acceptable, in order to reduce
to number of intersection tests that are performed I have decided to use the kdtree for the accelleration structure. The
construction of the kdtree is performed during the initialiseation of the triangle meshes as it is a static data structure, due
to it being a static data structure their is an advantage to performing more extensive computation in the construction in
order to create a higher quality of kdtree. TODO:Add sah and analysis.

\subsection{Intersection Tests (0.5 pages)}
Intersection tests are another vital part of any raytracer. a number of intersection tests are needed, I will not include
an extensive treatment of the intersection tests mearly a listing of the most notable used.

\begin{description}
\item[Triangle] Moller-Trumbore intersection test \cite{MolTru97}
\item[AABB]
\item[kD-tree] As described in TODO:
\end{description}

\subsection{volume intersection}
Unlike intersection with all other object types the intersection with a participating media is non deterministic, this is
due to the fact that the intersection of the ray is determined by the extinction coefficient which gives a probablilty
of interaction, the depth of the interaction is calculated with equation \todo{add equation}, performing this non-deterministic
intersection test allows for objects within the medium to be interacted with when performing ray-marching.

\missingfigure{Non-deterministic intersection test}
\subsection{Photon Mapping}
For each of the intersection points that are found as a result of the raytracing algorithm we must calculate the global
illumination at the point, this is performed with the photon maps that are associated with the scene.

\subsection{Nearest Neighbour search}
\todo{Find the correct location for this}
By far the most expensive operation in the photon mapping algorithm is the nearest neghbour search as we perform
many more of these operations per pixel, for radiance estimates that contain tens, hundreeds or thousands of photons
efficiently finding the $n$ closest photons is important.

\missingfigure{Nearest neighbour search}

\subsection{Sampling Strategy}
Many times within the code there is a need to sample from different distributions such as sampling from a hemisphere
of directions around a normal, the need for this can be seen from the definition of the radiance estimate given in
the desctription of the photon mapping algorithm as we will be evaluating the integral with monte carlo methods.

\subsection{Shading}
The intersection point of the primary ray is used as the point where we will evaluate the radiance for that will determin
the colour of the pixel, this is done by combining the result of direct lighting, indirect lighting, castics and if present
scattering and attenuation due to participating media, this is commonly expressed as a sum of integrals that sum to a solution
to the rendering equation, equation \todo{Add the equation} shows this.

\subsubsection{Direct Illumination}
Direct illumination can be calculated direclty by calculating the radiance incident to the intersection point for each of the
lights in the scene.

\missingfigure{Direct Illumination}

\subsubsection{Caustic Contribution}
Caustic illumintation indirect illumitionan from photons that take purly specular paths prior to being stored at a non-specular
surface.

\subsubsection{Multiple diffuse bounves and final gathering}

While it is possible to estimate the contribition to the radiance at an intersection point directly from the photon map this
approach can cause visual artifacts due to variance in the estimate in order to reduce these artifacts we perfom a final gather
stage at the point of intersect that produces a diffuse ray that is traced into the scene until a non-specular object is intesected,
we then perform the radiance at this point and use this information to estimate the radiance incident at the original point of intersection.
In order for the final gather to produce a correct estimate of the radiance we need to perform this stage multiple times per pixel, as we
are performing distributed raytracing this is a trivial addition. When calcualting the radiance for the final gather it has been shown \todo{cite}
that if the distance of the final gather point is lower than some threshold perfoming an additional diffuse bounce reduces errors at geometry
such as sharp corners where the radiance estimate can be inaccurate due to accounting for photons not truly at the surface.

\begin{figure}
\centering
\includegraphics[width=\textwidth]{./images/final_gather.png}
\label{fig:final_gather}
\caption{Final Gather}
\end{figure}

\subsection{Volume Contribution}
In the case of a ray that intersects a participating media before intersecting a surface we need to calculate three components to
the radiance along the path of the ray in the medium, these are single scattering direct illumination, multiple scattering (in-scattering)
and attenuation (out-scattering)

\subsubsection{Attenuation}
As a ray travels through a medium the radiance can be reduced due to out-scattering and absortion, this can be calculated by evaluating
the integral given in equation \todo{Add the equation}, as we are only considering homogeneous participating media this can be
simplified as the properties being integrated are constant and a closed form solution is possible.

\missingfigure{Attenuation Equation}

\subsubsection{Direct Illumination}
\subsubsection{Muliple Scattering}


\newpage
\chapter{Testing and Optimisations 3 pages}
Testing of the program was performed in two main ways, testing of the indidual functions and testing of the system as a whole.
certain functions designed to accelerate the raytracing need to be compared to a base implementation that while slower is
guaranteed to be correct, for instance a finding the closest intersection of a ray and mesh, the simplest way to do this
is to test the ray and every triangle in the mesh and record the closest intersection, this is simple to perform and
the scope for errors in the implementation is low, unfortuanatly this is a very expensive operation that requires $\Omega(n)$
time complexity, in this project the k-d tree is used to accelerate the test, this can be performed in $\Omega(log(n))$ time
complexity, the code to perform the intersection test for this data structure is much more complicated and as a result
errors in the implementation are more likely. Testing of these complicated functions was done by comparing the output
of the simple function with the faster implementation this can be used to ensure that the outputs match, it can also be
used to see if the new implementation does indeed increase the performance of the code. Figure~\ref{fig:testing_performance_comp}

\begin{figure}[h!]
\missingfigure{Performance Comparison}
\label{fig:testing_performance_comp}
\caption{Performance difference for triangle intersections}
\end{figure}

