\section{Photon Map Generation}
We described in Chapter~\ref{chap:design} that the photon generation stage of the back end would run on multiple threads
to utilise the computing power of the machine the system is being run on, this required certain descitions to be made to
allow for the multiple threads to correclty distribute the power of the light sources into the scene.

Each thread is responsible for producing a certain proportion of the photons in the scene, once a thread has processed all
of these photons it will signal to the main thread that it has finished by sending a flag to the output queue the thread also
updates a global light emission count include the photons that the thread emitted for each of the lights, a seperate count for
each of the counts is kept as the processing of the threads may not happen at the same time.

\subsection{Photon Emission}
In order to trace photons into the scene we first need a method of creating photons, this is implemented in the light 
defintion, each light object has a associated function that will produce a random photon from the light that is consitent with 
the type of light, for instance a point light will generate a photon with origin exactlu at the origin of the light source and
in a random direction, an area light produces a photon with origin on the area of the light with direction taken from a
cosine weighted hemisphere distribution in the direction of the normal of the light. Each photon that is emitted from the
light sources begin with the full power of the light source, this will be scaled by the number of emitted photons after
all photons have been gathered.

\subsection{Photon Tracing}
Once we have created a photon from a light source we now need to trace the photon through the scene and record the interactions
of the photon until it is absorbed or leaves the scene, the function that performes this function is \texttt{trace\_photon} the
declaration of this function is given below.

\texttt{int trace\_photon(scene\_t *scene, ray\_t *ray, int light, double power[3], bool specular, bool diffuse, bool specular\_only};

The parameters ray, light and power defines the photon properties, ray defines the origin and direction of the photon, light is an index
to the light from which this light was emited and the power contains the flux of the photon. specular and diffuse
define describe the path that the photons have taken prior to the call to \texttt{trace\_photon}

Photons are traced much like a ray in traditional raytracing, first we calculate the closest intersection point of the photon
once found the photon interacts with the surface of the object at the point of intersection this interaction can be specular 
reflection , transmission, diffuse reflection or absorbtion, at this point of intersection we must determine the path of the reflected
photon, the power of this photon is determined by the reflectance of the surface i.e a photon with white light that interacts with
a red surface will only reflect photons with power in the red component. Using the approach of scaling the photons will create a photon
map that correctly describes the distribution of power in the scene, unfortunatally this can lead to many photons with low power as they
are scaled at each interaction, it can also in the case of surfaces with both specualr and diffuse componenet lead to exponenetial number
of photons being produced as we need to create a reflected photon for both paths scaled apporoaprealty, as a result of these considerations
we have used a monte-carlo method that reflectes at most one photon
per surface interaction with full power, we ensure the correct result in the photon map by only reflecting the photon
with probability based on the reftance of the surface (i.e reflecting 50 full power photons as oppose to 100 full power photons)
we sample the reflectance of the surface by forming a cumulative distribution funciton of the materials reflectance coefficients.
Given the specular reflecive, transmissive and diffuse reflectance coefficents $\rho_{s}, \rho_{t}, \rho_{d}$
of the object we can create a cumulative distribution of photon emittion paths (Figure~\ref{fig:rr_dist}) 
we then take a uniform variable $\xi$ between 0 and 1 if the variable is within the distribution we reflect a photon, otherwise
the path of the photon is terminated. As we are using rgb values to store reflectance properties the
coeffients used to perform the sampling are the average of the three componenets, so as the distribution of the three
colour bands is correct we must also perform scaling of the power parameter that we use in the next \texttt{trace\_photon}
call, note that the scaling preserves the power of the photon and the reflectance properties of the surface.

\begin{figure}[h]
\includegraphics{./images/russian_roulette_distribution.png}
\caption{Russian roulette distribution}
\label{fig:rr_dist}
\end{figure}

At each non-specular interaction the photons are written to the output queue to be processed, this includes
the power of the photon, the incident angle of the photon to the surface and flags to indicate if the surface
has interacted with a diffuse or specular surface prior to this interaction, we do not store photon interactions
at specular surfaces as they do not provide us with information that can be used when estimating the radiance as
the probability of an incoming photon contributing to the specular reflectance is zero due to the delta functions
in the specular BRDF.

When creating the caustic photon map we are only concerned with those interactions with diffuse surfaces that have
taken a path that contains at least one specular bounce and no diffuse bounces (\textbf{LS$^+$DE}), as a result \texttt{trace\_photon}
has as input a flag that will indicate that any path that does not match this description should be discarded.

\subsubsection{Participating Media}
If the photon interacts with a participating medium the photon will begin a random walk inside the medium, the path of this walk
is determined by the properties of the medium, these being the scattering and absorbtion coefficient, the sum of which
is called the extinction coefficient, at each point in the random walk the photon is stored and then can either be
scattered or absorbed, the probability of being scattered is determined by the Albedo $\Lambda$

\begin{figure}
\centering
\includegraphics[width=0.5\textwidth]{./images/random_walk.png}
\caption{Random walk of photons through participating media}
\label{fig:random_walk}
\end{figure}

\begin{equation}
\Lambda = \frac{\sigma_s}{\sigma_e}
%\caption{Participating Media Albedo}
\end{equation}

Each step in the random walk continues for a distance until the next interaction with the medium is calculated, this distance
is calculated by importance samping the distance $\Delta x$ such that the expected distance is equal to that of the medium \cite{JensenBook},
the distance is given by:

\begin{equation}
\label{eq:volume_dist_importance}
\Delta x = \frac{-log(\xi)}{\sigma_t}
\end{equation}

where $\xi$ is a uniform random variable in the range [0, 1]. If a photon leaves a participating medium or intersects with an
object in the medium it is treated as if it was not in the medium and is stored in the global photon map.

\subsection{Photon Processing}
Each thread that traces the photons in the scene writes to a common queue, this queue is read by a single processing thread
that clasifies the photons by the path of the photon and stores the photons in one of three lists, the global, caustic and
volume photon list, this thread will continue to read photons from the queue until each of the threads have signaled that they
have completed their portion of the photons, this signal is in the form of a flag that is passed in the same queue as the
photon data. After all photons have been collected the power of the photons are scaled by the number of photons that were
emmited (note this is not the number of photons in the list but the total number emmited) from the same light as the
photon.

\subsubsection{K-D Tree Balancing}
When performing radiance estimations with the photon map we will be performing nearest neighbour searches on points within
the map, this requires the photon map to be arrainged in a manner such that this search can be performed efficiently.
In order to do this we store the photon map in a left-balanced k-d tree. A left balanced tree is a tree structure where
at each level of the tree the depth of the children differs by at most one,
this allows us to store the photon map in an array with the location of the children in the photon map known implicitly,
for a photon in position $i$ the children of the photon can be found at the $(2i + 1)^{th}$ and $(2i + 2)^{th}$
location for the left and right tree respectivly, the next stage is two transform the photon lists into a k-d tree.

\begin{algorithm}
\begin{algorithmic}
\caption{Balanced K-D tree construction}
\label{alg:balance}
\label{alg:balance}
\Function{balance}{photons, index, max\_index}
\If
{
photons.size \textgreater 1
}
{
	\State axis $\gets$ \Call{select\_axis}{photons}
	\State left, median, right $\gets$ \Call{partition\_around\_median}{photons, axis}
	\State left\_index  $\gets$ 2 * index + 1
	\State right\_index $\gets$ 2 * index + 2

	\If{left\_index \textless max\_index}
		\State left $\gets$ \Call{balance}{left, left\_index,max\_index}
	\EndIf

	\If{right\_index \textless max\_index}
		\State right $\gets$ \Call{balance}{right, right\_index, max\_index}
	\EndIf

	\Return{left + median + right}
}
\Else
{

	\Return{photons}
}
\EndIf
\EndFunction

\end{algorithmic}
\end{algorithm}

\begin{figure}
\centering
\begin{subfigure}{0.4\textwidth}
\includegraphics[scale=0.8]{./images/left-balanced-tree.png}
\caption{}
\end{subfigure}
\begin{subfigure}{0.4\textwidth}
\includegraphics[scale=0.8]{./images/non-left-balanced-tree.png}
\caption{}
\end{subfigure}
\caption{Left Balanced Tree}{(a) demonstrates a left balanced tree while (b) does not due to node F}
\end{figure}

\subsubsection{Selection Statistic}
In Algorithm~\ref{alg:balance} we partition the photons around the median of the list of photons, in order for
this partition to be stored in an array with no explicit node pointers we must choose the median position such
that the tree will be left balanced \cite{baerentzen03}, Algorithm~\ref{alg:bal-median} describes the procedure nessacerry for this
to be the case.

\begin{algorithm}
\caption{Balanced Median Calculation}
\label{alg:bal-median}
\begin{algorithmic}
\State $n \gets 0$
\While{$2^{n + 1} <= N$}
	\State n $\gets$ n + 1
\EndWhile
\State $M \gets 2^n$
\State $R \gets N - (M - 1)$
\If{$R <= M / 2$}
	\State \Return $(M - 2) / 2 + R$
\Else
	\State \Return $(M - 2) / 2 + M / 2$
\EndIf
\end{algorithmic}
\end{algorithm}
