\section{Participating Media}
In the case of a ray that intersects a participating media before intersecting a surface we need to calculate three components to
the radiance along the path of the ray in the medium, these are single scattering direct illumination, multiple scattering (in-scattering)
and attenuation (out-scattering)

\subsection{Ray Marching}
In order to evaluate the radiace from the participating media we perform a ray-march, this is an iterative operation that evaluates the
radiance along the ray as it moves through the participating media.

\begin{equation}
L(x, \omega) = \sum\limits_{i = 1}^N L_l(x, \omega_l')p(x, \omega_l', \omega)\sigma_s(x)\Delta x + e^{- \sigma_t \Delta x} L (x + \omega \Delta x, \omega)
\end{equation}

\subsection{Attenuation}
As a ray travels through a medium the radiance can be reduced due to out-scattering and absortion, this can be calculated by evaluating
the integral given in equation \todo{Add the equation}, as we are only considering homogeneous participating media this can be
simplified as the properties being integrated are constant and a closed form solution is possible.

\missingfigure{Attenuation Equation}

\subsection{Direct Illumination}
At each point in the ray march we also evaluate the contribution from each light in the scene due to single scattering, this is performed
by performing an additional ray march in the direction of the light and evalute the radiance arriving at the point on the ray.

\subsection{Muliple Scattering}
In order to evaluate the effect of multiple scattering within the participating media we use the photon map and the volume radiance estimate
for the photon map, as with direct illumination we evaluate the in-scattering at distctreate points along the path of the ray.
