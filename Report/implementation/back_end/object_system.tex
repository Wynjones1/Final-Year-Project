\todo{rephrase this introduction to the section}
In order to meet requirement \todo{find requirement to ref} the system needs to be able support arbitrary objects descriptions
that can be traced in the scene, this is implementatied as a structure that contains function pointers that perform functions
that must be supported by all raytracable objects and the data related to the individual object.
By encapsulating the data and the functions performed by the object it is possible to create algorithms on objects in the
abstract such that the details of the object can be ignored, for instance calculating a reflected ray at a point if intersection
is a function of only the surface normal at that point, wheither it is a triangle mesh or sphere the same ray will be calculated,
a full list of the functions that need to be defines is given in Figure~\ref{fig:object_funcs}

\begin{figure}[h]
\begin{description}
\item[int  (*intersection\_func)(object\_t *o, ray\_t *ray, intersection\_t *info)] \hfill \\
	Tests if the input ray intersectes with the object, returns the result and any other information is stored in the input variable info
\item[void (*bounds\_func)(object\_t *o, aabb\_t *bounds)] \hfill \\
	Returns the bounds of the object which is used during the scene-ray intersection test.
\item[void (*normal\_func)(object\_t *o, intersection\_t *info, double *normal)] \hfill \\
	Returns the normal of an object at a given intersection point defined in info.
\item[void (*tex\_func)(object\_t *o, intersection\_t *info, double *tex)] \hfill \\
	Returns 2-d texture coordinates in tex of the object at the point of intersection info.
\item[void (*delete\_func)(object\_t *o)] \hfill \\
	Frees any memory and other resources used by the object.
\item[void (*shade\_func)( object\_t *object, scene\_t *scene, intersection\_t *info)] \hfill \\
	Returns the colour at the surface of the object at the intersection point defined in info the result is stored in info.
\end{description}
\caption{Functional Definition of an Object}
\label{fig:object_funcs}
\end{figure}


We will now attempt to give an overview of the implementation of the objects currently used within the system.

\subsection{Sphere}
The sphere primitive is defined implicitly as an origin point and distance from the origin, including a sphere primitive is
a natural choice as many of the operations that we need to perform such as intersection tests and finding the normal at the
point of intersection is much simpler for spheres that most other surfaces allowing us to use this primitive for testing
changes while being confident that any errors that we find are not within the object definition of the sphere, the same cannot
be said for a mesh where the optimised intersection test used is fairly complex and contains more scope for implementational errors.

\subsubsection{Intersection test}
Given that we define a ray and a sphere in a parametric form it is intuitive to solve the intersection of these objects implicity,
given a ray defined $\vec{o} + t \vec{d}$ and the equation for a sphere $(p - c) \cdot (p - c) = r^2$ where p is a point on the the
sphere, c the center of the sphere and r the radius of the sphere, combining these two equations we get:

\begin{equation*}
(\vec{o} + t\vec{d} -\vec{c}) \cdot (\vec{o} + t\vec{d} - \vec{c}) = r^2
\end{equation*}

expanding this gives us:

\begin{equation*}
(\vec{d} \cdot \vec{d})t^2 + 2(\vec{o} - \vec{c}) \cdot \vec{d}t + (\vec{o} - \vec{c}) \cdot (\vec{o} - \vec{c}) - r^2 = 0
\end{equation*}

this can be solved as a quadratic equation of the form $At^2 + Bt + C = 0$ where:

\begin{align*}
A &= \vec{d} \cdot \vec{d}\\
B &= 2(\vec{o} - \vec{c}) \cdot \vec{d}\\
C &= (\vec{o} - \vec{c}) \cdot (\vec{o} - \vec{c}) - r^2\\
\end{align*}

as this is a quadratic equation we have two possible solutions for t given by:

\begin{align*}
t_0 = \frac{- B - \sqrt{B^2 - 4AC}}{2A}
,
t_1 = \frac{- B + \sqrt{B^2 - 4AC}}{2A}
\end{align*}

for the case when $B^2 - 4AC < 0$ we have no real solutions and the ray does not intersect with the sphere, otherwise we
take the smallest positive intersection value, if both values are negative we do not intersect the ray.

\subsection{Mesh}
The mesh primitive is defined as a list of vertices and indices that define the triangles from the list of vertices, also
stored are the surface normals for the triangles and if included the texture coordinates. We will not discuss the implementation
for reading in the mesh in detail as it is a rather simple function that reads from a PLY file and extracts the data.

\todo{Add the structure definition of the mesh type}

\subsubsection{K-D Tree Construction}
\label{sec:obj_mesh}
Intersecting a triangle mesh with a bruteforce method would be prohibitivly expensive for a mesh with many triangles, it
is common then, to use a acceleration struction that reduces the number of intersection. The
construction of the kdtree is performed during the initialiseation of the triangle meshes as it is a static data structure.
We begin the contruction of the kd-tree with all
triangles in a single list, we then calculate an axis and position on that axis to seperate the triangles, this is calculated
by finding the median spread of the geometry in the voxel we then sort each of the triangles into two child lists, if a triangle is
to the left of the splitting plane it will be placed in the left childs triangle list and visa-versa for the right, if the
triangle spans the split plane it will be added to both lists, we then call the tree building function recursivly on the
left and right child and delete the list for this node, we terminate when a list contains less than a threshold number
of triangles in its bucket or a maximum depth is reached or if all triangles are placed in a single node.

\subsubsection{Intersection test}
Given the k-d tree that we have constructed for a mesh we can now describe an efficient top-down intersection algorithm,
we begin at the root of the k-d tree, given that the tree has been split along an axis we can calculate which of the
child voxels are closer to the ray by calculating the values of intersection for the two voxel, if the ray origin is
within a voxel it is set to be the near voxel and the other child the far, otherwise we compare the t values for the
voxel intersection. Once we know which voxel to traverse, if the ray exits the near voxel before crossing the splitting
plane we do not need to test the far voxel as we cannot intersect any geometry in that voxel, we perform this procedure
recursibly on the near and possibly far voxel until we reach a leaf node, we then test all triangles in the index list,
if we find an intersection we record the t parameter for the intersection and check the rest of the list, if an intersection
if found we can return the intersection point and discard any other voxels that we were going to check as they cannot
have an intersection closer. There is a subtle point that must be noted, when we are checking for that intersection of
a triangle in a list if the intersection point is found to be more that the exit point of the voxel then we do not
record this intersection as the triangle spans voxels and we need to check for closer intersection.

\todo{Add a visual representation of the k-d tree intersection test, specifically the case where the triangle spans voxels}


\subsection{Material Properties}
Each object has associated with it a material property that defines the reflectance properties of the material.
The system currenlty suports diffuse and pure specular components each being defined as a three floating point
values for red, green and blue compnonets respectivly, texture mapping is also supported for the diffuse component,
for specular transmission an index of refraction is defined that determins the direction of refracted rays through the
material. We also store the average values for these components explicitly as they are frequentlty used when performing
russian roulette to evaluate integrals during photon map construction and raytracing. Participating media require additional
components for the scattering and absorption coefficients as discussed in the Literature Survey, these are also stored
as rgb components, we also store derived properties, namely the extinction coefficent and albedo of the matrial.
