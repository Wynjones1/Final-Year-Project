\section{Object System}
In order to meet requirement \todo(find requirement to ref) the system needs to be able support arbitrary objects descriptions
that can be traced in the scene, this is implementatied as a structure that contains function pointers that perform functions
that must be supported by all raytracable objects and the data related to the individual object.

\begin{description}
\item[normal] returns the normal at a point of intersection.
\item[bounds] return the bounds for the object.
\item[texture] return the texture coordinates at a point of intersection
\item[intersection] returns if a ray intersects a given object.
\end{description}

By encapsulating the data and the functions performed by the object it is possible to create algorithms on objects in the
abstract such that the details of the object can be ignored, for instance calculating a reflected ray at a point if intersection
is a function of only the surface normal at that point, wheither it is a triangle mesh or sphere the same ray will be calculated.

\subsection{Mesh}
\subsection{Sphere}

\subsection{Participating Media}

\subsection{K-D Tree Construction}
In order to satisy requirement TODO:Add the requirement bruteforce methods for raytracing are not acceptable, in order to reduce
to number of intersection tests that are performed I have decided to use the kdtree for the accelleration structure. The
construction of the kdtree is performed during the initialiseation of the triangle meshes as it is a static data structure, due
to it being a static data structure their is an advantage to performing more extensive computation in the construction in
order to create a higher quality of kdtree. TODO:Add sah and analysis. The construction of the tree is 

\subsection{Material Properties}
