Testing of the program was performed in two main ways, testing of the individual functions and testing of the system as a whole.
Certain functions designed to accelerate the ray-tracing need to be compared to a base implementation that while slower is
guaranteed to be correct, for instance a finding the closest intersection of a ray and mesh, the simplest way to do this
is to test the ray and every triangle in the mesh and record the closest intersection, this is simple to perform and
the scope for errors in the implementation is low, unfortunately this is a very expensive operation that requires $O(n)$
time complexity. In this project the k-d tree is used to accelerate the test, this can be performed in $O(log(n))$ time
complexity, the code to perform the intersection test for this data structure is much more complicated and as a result
errors in the implementation are more likely. Testing of these complicated functions was done by comparing the output
of the simple function with that of the faster implementation for identical data, it can also be
used to see if the new implementation does indeed increase the performance of the code. Figure~\ref{fig:testing_performance_comp}.
For other aspects of the system full testing was not possible due to time constraints, in these cases we used a heuristic
approach of testing whereby we ran the system with test scenes and compared the results of to out expectation of the final image.

\section{System Testing}

\subsection{Photon Viewer}
As the development of the system was performed roughly in the order described in the document, the photon map was produced before the
system used it in the radiance estimate when creating the images, as a concequence verifiying the results of the photon
map generator posed a particular challenge as it typically contained thousands of photons in a geometry independent data structure, as
part of the development of the product we created a companion utility to the main system that allows for the visualisation of the
photon map, when inputted into the program we can see if the results match the expectation of the distribution of the photons for a
given scene. The photon viewer is reads in data from a simple format containing as a first line the number of photons and then
a list of the same length of photon positions, powers and incident directions. This tool proved to be of great value also being used
to test the sampling functions. The design of the photon viewer is similar to that of the main GUI, differing in that it displays points
stored in an OpenGL display list \cite{khronos:2014:online}  (used to lower the number of draw calls) as oppose to rendering a
texture of the output image as in the main GUI.

\begin{figure}
\centering
\includegraphics[width=0.7\textwidth]{./images/photon_viewer.png}
\label{fig:photon_viewer}
\caption{Photon Viewer Running}
\end{figure}

\subsection{Pixel Tracing}
When performing tests on the system it would frequently be the case that a change introduced obvious error in the output image,
it became apparent early in the development that being able to trace a single pixels path through the scene would make the
task of debugging that much easier. This functionality is available as the \texttt{--trace\_pixel} option.

\section{Performance Testing}
During the design and implementation of the system we have stated that certain decisions were made in order to improve the performance
of the system, in this section we will present results of testing that was performed in order to evaluate the gains from these decisions
in a concrete way.

\subsection{K-D tree intersection acceleration}
We stated in Chapter~\ref{chap:implementation} that we used triangle k-d trees to accelerate the intersection test for
meshes, this test is designed to test the gains that we saw when using the k-d tree, the test was performed by running the
system on a scene with a complex model (the Stanford dragon) with and without the k-d tree intersection test, for the case
without the k-d tree the intersection test is performed by testing each triangle in the mesh and recording the closest intersection.
For this test we have disabled photon mapping so as to only test the ray-tracing part of the system, this was done through
prepossessing defines in the system at compile time.

\subsection{Multicore Scaling}
In Chapter~\ref{chap:reqs} and \ref{chap:design} we highlighted the inherent parallelism possible in the
photon mapping algorithm in both the photon generation and ray-tracing stages, in order to
test the performance of the system and its ability to scale with more resources we decided to
acquire access to Amazon EC2 compute clusters, allowing us to run the system on up to 32 Intel
Xeon E5-2670 cores and 60GB of RAM \cite{amazon-instances}, we tested the system with various
settings such as resolution and number of photons in the photon map and in the radiance estimate
, each configuration will be run multiple times in order to reduce variance due to the hardware.
We shall test two parts of the system in isolation to gauge how well they have been parralised
the first is the photon map generation and the second the ray-tracing stage, in order to test them
in isolation we modify the system such that in the first case it exits after creating the photon
map and in the second the photon maps are not created at all.
